\subsection{Environment checks}
\label{sec:EnvChecks}

Given the heterogeneity of available systems, \GASPI{} provides
routines to control the environment in which the application will
run. These include test of the running software and hardware
environment and definition or retrieval of configuration options.

These routines allow the verification of the environment prior to the
initialization (\gaspifunction{start}) phase.

\subsubsection{\gaspifunction{get\_hostname}}

For some environmental checks it is required to provide the host name
of a node, without any reference to ranks or groups. It is
\gaspisemantic{synchronous} 
\gaspisemantic{local} \gaspisemantic{blocking} procedure.
It is a procedure
which can be invoked in any of the \GASPI{} execution phases.

\begin{FDef}
\begin{FDefSign}
\begin{verbatim}
gaspi_return_t
gaspi_get_hostname (gaspi_num_t host_num
                   , gaspi_hostname_t hostname)
\end{verbatim}
\end{FDefSign}
\parameterlistbegin
\parameterlistitem{in}{hostnum}{The number of the node to find the host name }
\parameterlistitem{out}{hostname}{The host name of the rank}
\parameterlistend
\FStdRetDescNOTimeout
\end{FDef}

The procedure \gaspifunction{get\_hostname} retrieves the hostname of
the node with number \parameter{host\_num} and places it at
the \parameter{hostname} output parameter.

In case of success, i.e. \GASPISUCC{}, the \parameter{hostname} holds
the host name used to identify the node.

In case of error, for example an invalid rank number, the return value
is \GASPIGERR{}.

The procedure can be invoked in any of the \GASPI{} execution phases.

\subsubsection{\gaspifunction{get\_num\_hosts}}

For some environmental checks it is required to provide the host name
of a node, without any reference to ranks or groups. It is
\gaspisemantic{synchronous} 
\gaspisemantic{local} \gaspisemantic{blocking} procedure.

\begin{FDef}
\begin{FDefSign}
\begin{verbatim}
gaspi_return_t
gaspi_get_num_hosts ( gaspi_num_t num_hosts )
\end{verbatim}
\end{FDefSign}
\parameterlistbegin
\parameterlistitem{out}{num\_hosts}{The number of hosts }
\parameterlistend
\FStdRetDescNOTimeout
\end{FDef}

The procedure \gaspifunction{get\_hostname} retrieves the number of
hosts and places it at
the \parameter{num\_hosts} output parameter.

In case of success, i.e. \GASPISUCC{}, the \parameter{num\_hosts} holds
the number of nodes visible to \GASPI{}.

In case of error, for example an invalid rank number, the return value
is \GASPIGERR{}.

The procedure can be invoked in any of the \GASPI{} execution phases.

\subsubsection{\gaspifunction{check\_shared\_libraries}}

The \gaspifunction{check\_shared\_libraries} procedure is a
\gaspisemantic{synchronous} 
\gaspisemantic{local} \gaspisemantic{time-based blocking} procedure which checks if all shared
libraries required by the application are found. The application runs
on a clean environment and therefore some libraries might not be found
as in the master node. This procedure ensures that everything is OK.
It is supposed, to be run before the actual \GASPI{} start-up.
It is a procedure
which can be invoked in any of the \GASPI{} execution phases.

\begin{FDef}
\begin{FDefSign}
\begin{verbatim}
gaspi_return_t
gaspi_check_shared_libraries ( gaspi_hostname_t hostname
                             , gaspi_timeout_t timeout
                             )
\end{verbatim}
\end{FDefSign}
\parameterlistbegin
\parameterlistitem{in}{hostname}{the host name where to run the check}
\parameterlistitem{in}{timeout}{the timeout}
\parameterlistend
\FStdRetDesc
\end{FDef}

After successful procedure completion, i.e. return value \GASPISUCC{}
all shared libraries have been found on the node \parameter{hostname}. 

In case of error, the return value is \GASPIGERR{}. 

In case of timeout, the return value is \GASPITIME{}.

The procedure can be invoked in any of the \GASPI{} execution phases (c.f. \secref{SubSec:ExecPhases} ).

\subsubsection{\gaspifunction{check\_port}}

The \gaspifunction{check\_port} is a \gaspisemantic{synchronous} 
\gaspisemantic{local} \gaspisemantic{time-based blocking}
procedure which checks if the given port number for remote program
start-up is available to use. It is a procedure
which can be invoked in any of the \GASPI{} execution phases.

\begin{FDef}
\begin{FDefSign}
\begin{verbatim}
gaspi_return_t
gaspi_check_port ( gaspi_hostname_t hostname
                 , gaspi_port_t port
                 , gaspi_timeout_t timeout
                 )
\end{verbatim}
\end{FDefSign}
\parameterlistbegin
\parameterlistitem{in}{hostname}{The host name where to run the check}
\parameterlistitem{in}{port}{the port number to check}
\parameterlistitem{in}{timeout}{the timeout}
\parameterlistend
\FStdRetDesc
\end{FDef}

After successful procedure completion, i.e. return value \GASPISUCC{}
the \parameter{port} is available on node \parameter{hostname} for
remote program start-up.

In case of error, the return value is \GASPIGERR{}. 

In case of timeout, the return value is \GASPITIME{}.

The procedure can be invoked in any of the \GASPI{} execution phases (c.f. \secref{SubSec:ExecPhases} ).

\subsubsection{\gaspifunction{process\_find}}

The \gaspifunction{process\_find} procedure is a \gaspisemantic{synchronous} 
\gaspisemantic{non-local} \gaspisemantic{time-based blocking} procedure checks if there
is a running \GASPI{} binary running on a given host.
The procedure can be invoked in any of the \GASPI{} execution phases. 

\begin{FDef}
\begin{FDefSign}
\begin{verbatim}
gaspi_return_t
gaspi_process_find ( gaspi_hostname_t hostname
                   , gaspi_bool_t running_process
                   , gaspi_timeout_t timeout
                   )
\end{verbatim}
\end{FDefSign}
\parameterlistbegin
\parameterlistitem{in}{hostname}{the host name where to run the check}
\parameterlistitem{out}{running\_process}{running process boolean}
\parameterlistitem{in}{timeout}{the timeout}
\parameterlistend
\FStdRetDesc
\end{FDef}

With a successful procedure completion, i.e. return value
\GASPISUCC{}, the remote \GASPI{} process check has been 
completed successfully and the parameter \parameter{running\_process}
indicates whether there is a running process or not.

In case of error, the return value is \GASPIGERR{} and the output parameter
\parameter{running\_process} has an undefined value.

In case of timeout, the return value is \GASPITIME{} and the output parameter
\parameter{running\_process} has an undefined value.



\subsubsection{\gaspifunction{check\_network}}

The \gaspifunction{check\_network} procedure is a \gaspisemantic{synchronous} 
\gaspisemantic{local} \gaspisemantic{time-based blocking} procedure which checks if the network is
running properly. It is a procedure
which can be invoked in any of the \GASPI{} execution phases.

\begin{FDef}
\begin{FDefSign}
\begin{verbatim}
gaspi_return_t
gaspi_check_network ( gaspi_hostname_t hostname
                    , gaspi_network_t network
                    , gaspi_timeout_t timeout
                    )
\end{verbatim}
\end{FDefSign}
\parameterlistbegin
\parameterlistitem{in}{hostname}{the host name where to run the check}
\parameterlistitem{in}{network}{the network infrastructure to check}
\parameterlistitem{in}{timeout}{the timeout}
\parameterlistend
\FStdRetDesc
\end{FDef}

With a successful procedure completion, i.e. return value
\GASPISUCC{}, there is certainty that the \parameter{network}
infrastructure is up and running.

In case of error, the return value is \GASPIGERR{}. 

In case of timeout, the return value is \GASPITIME{}.

\subsubsection{Example}

Setup phase environment check example.