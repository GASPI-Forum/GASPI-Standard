\documentclass[plainarticle,german,final,hyperref,utf8]{article}
\usepackage{setspace}
\usepackage{listings}
\usepackage{graphicx}
\usepackage{color}

\author{Olaf Krzikalla}
\title{Implementing a statistical counter with pGaspi - a Tutorial}

\definecolor{dkgreen}{rgb}{0,0.6,0}
\definecolor{mauve}{rgb}{0.58,0,0.82}
\definecolor{colorkeyword}{rgb}{0.1,0.1,0.6}
\definecolor{colorassert}{rgb}{0.5,0.5,0.5}


\lstset{
basicstyle=\ttfamily
, frame=single
, numbers=left
, language=C
, numberstyle=\tiny
, stepnumber=1
, showstringspaces=true
, keepspaces=true
, columns=fullflexible
, classoffset=0
, morekeywords={gaspi_rank_t,gaspi_barrier,gaspi_timeout_t,gaspi_return_t,gaspi_config_t,gaspi_proc_init,gaspi_proc_term,gaspi_proc_kill
  ,gaspi_group_commit,gaspi_group_create,gaspi_group_add,gaspi_group_size,gaspi_group_ranks
  ,gaspi_segment_id_t,gaspi_segment_alloc,gaspi_segment_register,gaspi_segment_create
  ,gaspi_segment_ptr,gaspi_queue_id_t,gaspi_offset_t,gaspi_socket_id_t,gaspi_size_t,gaspi_pointer_t
  ,gaspi_write,gaspi_read,gaspi_wait
  ,gaspi_notify,gaspi_notify_reset,gaspi_notify_wait ,gaspi_atomic_value_t
  ,gaspi_proc_num,gaspi_proc_rank
  ,gaspi_number_t,gaspi_notification_t,gaspi_notification_id_t, gaspi_time_t
  ,gaspi_atomic_compare_swap,gaspi_atomic_set
  ,GASPI_SUCCESS,GASPI_TIMEOUT,GASPI_ERROR,GASPI_GROUP_ALL
  ,GASPI_BLOCK,GASPI_TEST,GASPI_CONFIGURATION_DEFAULT,GASPI_NORANK,GASPI_NOSTRING,GASPI_ALLOC_DEFAULT,
  ,gaspi_notify_waitsome,gaspi_notify_size,gaspi_error_message
  ,gaspi_queue_size_max,gaspi_queue_size,gaspi_time_get
  ,gaspi_network_t, gaspi_number_t
  }
, commentstyle=\color{dkgreen}
, keywordstyle=\color{colorkeyword}
, classoffset=1
, morekeywords={ASSERT}
, keywordstyle=\color{colorassert}
, classoffset=0
}

\newcommand{\insertlisting}[2]{\lstset{
basicstyle=\ttfamily
, caption={#2}
, frame=single
, numbers=left
, numberstyle=\tiny
, stepnumber=1
, showstringspaces=true
, keepspaces=true
, columns=fullflexible
, classoffset=0
, morekeywords={gaspi_rank_t,gaspi_barrier,gaspi_timeout_t,gaspi_return_t,gaspi_config_t,gaspi_proc_init,gaspi_proc_term,gaspi_proc_kill
  ,gaspi_group_commit,gaspi_group_create,gaspi_group_add,gaspi_group_size,gaspi_group_ranks
  ,gaspi_segment_id_t,gaspi_segment_alloc,gaspi_segment_register,gaspi_segment_create
  ,gaspi_segment_ptr,gaspi_queue_id_t,gaspi_offset_t,gaspi_socket_id_t,gaspi_size_t,gaspi_pointer_t
  ,gaspi_write,gaspi_read,gaspi_wait
  ,gaspi_notify,gaspi_notify_reset,gaspi_notify_wait ,gaspi_atomic_value_t
  ,gaspi_proc_num,gaspi_proc_rank
  ,gaspi_number_t,gaspi_notification_t,gaspi_notification_id_t, gaspi_time_t
  ,gaspi_atomic_compare_swap,gaspi_atomic_set
  ,GASPI_SUCCESS,GASPI_TIMEOUT,GASPI_ERROR,GASPI_GROUP_ALL
  ,GASPI_BLOCK,GASPI_TEST,GASPI_CONFIGURATION_DEFAULT,GASPI_NORANK,GASPI_NOSTRING
  ,gaspi_notify_waitsome,gaspi_notify_size,gaspi_error_message
  ,gaspi_queue_size_max,gaspi_queue_size,gaspi_time_get
  ,gaspi_network_t, gaspi_number_t
  }
, commentstyle=\color{dkgreen}
, keywordstyle=\color{colorkeyword}
, classoffset=1
, morekeywords={ASSERT}
, keywordstyle=\color{colorassert}
, classoffset=0
}\lstinputlisting{#1}}


\newcommand{\GASPI}{{\sc Gaspi}}
\newcommand{\PGASPI}{{\tt pgaspi}}

\begin{document}

\section*{Implementing a statistical counter in \GASPI{}}

This tutorial shows you how statistical counters can be incorporated in a \GASPI{} application using the \PGASPI{} interface.
It is based on the file {\tt tutorial\_profiler.c}, which comes along with this document. 
There are a lot of comments in {\tt tutorial\_profiler.c} explaining the actual implementation of the profiler step by step.   
  
The \PGASPI{} interface enables you to implement hooks for gaspi functions by exploiting the dynamic linking capabilites of the operating system.
Thus in a first step you have to ensure that your \GASPI{} application is dynamically linked against your \GASPI{} library.
If you use GPI-2, you can build a shared object of the non-debug version by using the following all-clause in GPI-2/src/Makefile:   

\begin{lstlisting} 
all:
  gcc -O2 -fPIC -I$(OFED_PATH)/include -D_GNU_SOURCE -c GPI2.c
  gcc -shared -Wl,-soname,libGPI2.so GPI2.o -o libGPI2.so
  cp libGPI2.so ../lib64
  cp GASPI.h ../include
\end{lstlisting} 

You can check whether your \GASPI{} application uses a shared library using {\tt ldd}.
The actual profiler wrapper is also build as a shared library.

\begin{lstlisting} 
gcc tutorial_profiler.c -fPIC -shared -Idir/to/GASPI/include 
    -o tutorial_profiler.so  
\end{lstlisting} 

Before you can run and profile your application, you have to tell your operating system, that {\tt tutorial\_profiler.so} wraps your native \GASPI{} library. 
In Linux you do this by setting the {\tt LD\_PRELOAD} environment variable.
  
\begin{lstlisting} 
export LD_PRELOAD=/dir/to/profiler/tutorial_profiler.so
\end{lstlisting} 

You have to ensure that this environment variable is visible on every node. 
One simple way to ensure this is to add this line to your {\tt .bashrc}.

Finally you can run your \GASPI{} application. 
If you use the profiler code of this tutorial, it should tell you the total numbers of bytes read and written:

\begin{lstlisting} 
Number of bytes written by this rank (rank 0): 0
Number of bytes read by this rank (rank 0): 13088
\end{lstlisting} 

%\pagebreak
%\appendix
%\insertlisting{tutorial_profiler.c}{}


\end{document}
